% THIS IS SIGPROC-SP.TEX - VERSION 3.1
% WORKS WITH V3.2SP OF ACM_PROC_ARTICLE-SP.CLS
% APRIL 2009
%
% It is an example file showing how to use the 'acm_proc_article-sp.cls' V3.2SP
% LaTeX2e document class file for Conference Proceedings submissions.
% ----------------------------------------------------------------------------------------------------------------
% This .tex file (and associated .cls V3.2SP) *DOES NOT* produce:
%       1) The Permission Statement
%       2) The Conference (location) Info information
%       3) The Copyright Line with ACM data
%       4) Page numbering
% ---------------------------------------------------------------------------------------------------------------
% It is an example which *does* use the .bib file (from which the .bbl file
% is produced).
% REMEMBER HOWEVER: After having produced the .bbl file,
% and prior to final submission,
% you need to 'insert'  your .bbl file into your source .tex file so as to provide
% ONE 'self-contained' source file.
%
% Questions regarding SIGS should be sent to
% Adrienne Griscti ---> griscti@acm.org
%
% Questions/suggestions regarding the guidelines, .tex and .cls files, etc. to
% Gerald Murray ---> murray@hq.acm.org
%
% For tracking purposes - this is V3.1SP - APRIL 2009

\documentclass{style}

\usepackage{paralist}

\begin{document}

\title{Cache Oblivious B-trees}
%
% You need the command \numberofauthors to handle the 'placement
% and alignment' of the authors beneath the title.
%
% For aesthetic reasons, we recommend 'three authors at a time'
% i.e. three 'name/affiliation blocks' be placed beneath the title.
%
% NOTE: You are NOT restricted in how many 'rows' of
% "name/affiliations" may appear. We just ask that you restrict
% the number of 'columns' to three.
%
% Because of the available 'opening page real-estate'
% we ask you to refrain from putting more than six authors
% (two rows with three columns) beneath the article title.
% More than six makes the first-page appear very cluttered indeed.
%
% Use the \alignauthor commands to handle the names
% and affiliations for an 'aesthetic maximum' of six authors.
% Add names, affiliations, addresses for
% the seventh etc. author(s) as the argument for the
% \additionalauthors command.
% These 'additional authors' will be output/set for you
% without further effort on your part as the last section in
% the body of your article BEFORE References or any Appendices.

\numberofauthors{3} %  in this sample file, there are a *total*
% of EIGHT authors. SIX appear on the 'first-page' (for formatting
% reasons) and the remaining two appear in the \additionalauthors section.
%
\author{
% You can go ahead and credit any number of authors here,
% e.g. one 'row of three' or two rows (consisting of one row of three
% and a second row of one, two or three).
%
% The command \alignauthor (no curly braces needed) should
% precede each author name, affiliation/snail-mail address and
% e-mail address. Additionally, tag each line of
% affiliation/address with \affaddr, and tag the
% e-mail address with \email.
%
% 1st. author
\alignauthor
Salman Ahmad\\
\affaddr{MIT CSAIL}
\email{saahmad@mit.edu}
% 2nd. author
\alignauthor
Leilani Battle\\
\affaddr{MIT CSAIL}
\email{leibatt@mit.edu}
% 3rd. author
\alignauthor
Stephen Tu\\
\affaddr{MIT CSAIL}
\email{stephent@mit.edu}
}
% There's nothing stopping you putting the seventh, eighth, etc.
% author on the opening page (as the 'third row') but we ask,
% for aesthetic reasons that you place these 'additional authors'
% in the \additional authors block, viz.
\date{7 December 2011}
% Just remember to make sure that the TOTAL number of authors
% is the number that will appear on the first page PLUS the
% number that will appear in the \additionalauthors section.

\newcommand{\lhyperceil}{\lceil\lceil}
\newcommand{\rhyperceil}{\rceil\rceil}
\newcommand{\lhyperfloor}{\lfloor\lfloor}
\newcommand{\rhyperfloor}{\rfloor\rfloor}

\newcommand{\Search}{\textsc{Search(k)}}
\newcommand{\Insert}{\textsc{Insert(k, v)}}
\newcommand{\Scan}{\textsc{Scan(a, b)}}

\maketitle
\begin{abstract}
Fill me in
\end{abstract}

\section{Introduction}

\cite{BenderFaFi07}

\subsection{External Memory Algorithms}

\subsection{B-trees}

\subsection{CO vs CA}

\subsection{CO B-trees}

\section{Background}

\subsection{Original CO B-trees}

\subsection{Structure}

\subsection{Layout}

\subsection{Complexity}

\subsection{Operations}

\section{Variable Length Strings}

\subsection{Description}

\subsection{Operators}

\subsection{Complexity}

\section{Streaming B-trees}

\subsection{Description}

\subsection{Operators}

\subsection{Complexity}

\section{Concurrency}

\subsection{Motivation}
B-trees are often used as an indexing data structure for various
database and file systems. Thus, for a cache-oblivious B-tree data structure
to be of practical value, it must be able to handle concurrent access and modification
of its internal state. The straw-man solution is to simply ensure mutual exclusion
on the entire data structure; this approach, however, yields poor performance.

Various concurrency schemes for traditional cache-aware B-trees have been
proposed \cite{BayerS77, LehmanY81}. We now survey a concurrency scheme for CO
B-trees, first presented in \cite{BenderFiGi05}. Bender develops two different
data structures to solve this problem: the first is an exponential CO B-tree,
the second is a CO B-tree based on a packed memory array (PMA). Bender
also presents a lock-free variant of the latter data structure. For
the remainder of the discussion, assume that the keys are fixed-size integer
keys.

\subsection{Description}
Before describing the data structures in more detail, we first consider
the concurrency model which is used throughout \cite{BenderFiGi05}.

\subsubsection{Concurrency Model}
The concurrency model starts with a system that has a total cache
size of $M$ and block size of $B$. There are $P$ processors in the
system, each with a cache of size $\frac{M}{P}$. A particular block
can reside in multiple caches in \textit{shared} mode. However, if a
processor wants to perform a write to a block, it must acquire it in 
\textit{exclusive} mode, in which case that block can only reside in
a single cache. In this model, a request for a block costs a single
memory transfer.

For concurrency control, two primitives are used. The first is
locks. The second is \textit{load-linked/store-conditional} (LL/SC).
The authors argue that LL/SC provides better performance than 
using a \textit{compare-and-swap} (CAS) primitive, but their 
techniques can be extended to use CAS.

\subsection{Exponential CO B-tree}
This data structure is based on a strongly weight-balanced 
exponential tree found in \cite{Bender2002}. A tuning parameter
$1 < \alpha < 2$ is chosen for the data structure. Each node
in the tree maintains both keys which divide up its children,
in addition to a \textit{right-link} pointer, which contains
both a pointer to a node's right sibling, plus the key
value of the minimum key in the node's right sibling. If no such
sibling exists, a sentinel $\infty$ pointer is used.

A node's layout in memory is as follows. Suppose a node has $k$
elements in it. The first part of the node consists of a static
CO search B-tree, containing the $k$ elements. The tree is sized
to be a complete $\lhyperceil k \rhyperceil$-leaf tree. The second
part of the node consists of a $\lhyperceil k \rhyperceil$-size array
contained the $k$ elements in sorted order. The leaf nodes of the
CO search tree point to the associated entries in the array.

\subsubsection{Operations}
The operations supported by the exponential CO B-tree are \Search{}
and \Insert{}. The authors do not discuss range-queries, and this
is presumably due to the concurrency issues in implementing a fully-consistent
range scan (weakly-consistent range scans could be implemented in the data
structure with very little modification), although interestingly enough
the PMA variant of B-tree (discussed in Section \ref{sec:pma}) does
cover range scans. 

\Search{} works as follows. For a non-leaf node, first check the
\textit{right-link} key. If the \textit{right-link} key is less than
$k$, then traverse the right link, and recurse. Otherwise, follow
the appropriate child pointer. When a leaf node is reached, only
follow \textit{right-link} pointers, until either $k$ has been located
or some element greater than $k$ is reached. No locks are acquired
during the entire search operation.

\Insert{} works as follows. First, \Search{} is used to locate the leaf node
where $k$ will be inserted into. A lock is acquired on that node, $k$ is
inserted, and the lock is released. Unlike a standard B-tree, there is no
maximum number of elements in a node; instead we split nodes probabilistically.
With probability ${1}/{2^{\alpha^h}}$, a node at height $h$ is split (leaf
nodes have $h = 0$). A split works as follows.  We reacquire a lock on the node
to be split. We split the node into $u$ and $u'$, where $u$ contains all the
keys $< k$, and $u'$ contains the keys $\geq k$. We release the lock, and
recursively insert $k$ into the parent of the split node. It is important
to note that, since $u$ contains a prefix of the original node to be split,
that we reuse the original node in constructing $u$ (that way, concurrent
searches can still locate the keys in $u$).

%TODO: some section about linearizability?

\subsubsection{Complexity}
Sequential search takes $O(\log_B{N} + \log_{\alpha}{\log_2{B}})$ block
transfers (with high probability). The probabilistic bound comes from the
probabilistic key promotion property. The intuition behind the proof of this
bound comes from the argument that when the tree has height $h =
\Omega(\log_{\alpha}{\log_2{\log_2{N}}})$, then the total number of keys in the
tree is $O(2^{\alpha^h})$. The proof is then completed by summing over the
range of $h$ which is probabilistic $O(\log_{\alpha}{\log_2{N}})$, the access
time from some level $h$ to find the correct child.

Sequential insert takes $O(\log_B{N} + \log{\alpha}{\log_2{B}})$ block
transfers (in expectation). This proof is constructed in a similar 
fashion as the search proof: consider inserting a tree at some height $h$.
By noting that the probability a particular node reaches height $h$
is $2^{-(\alpha^h-1)/(\alpha-1)}$, a summation over all levels
of the cost of rebuilding a node yields the time-bound.

\subsection{Packed Memory Array CO B-tree}
\label{sec:pma}
The PMA CO B-tree is a two level data structure, containing a static
CO B-tree which indexes into a packed-memory array. Bender introduces
a new packed-memory array, called a \textit{one-way packed-memory array},
which they argue is easier to manage concurrently.

\subsubsection{One-Way PMA Primer}

A one-way packed-memory array is a contiguous memory structure which has three
distinct regions: a leftmost region of empty space, an \textit{active region}
which contains the elements of the array, and a rightmost region of empty
space. The array only grows into the rightmost region, and shrinks from the
leftmost region (hence one-way).  The PMA is of size $m = \Theta(N)$, subject
to $m$ being a power of $2$, where $N$ is the number of elements.

When the \textit{active region} becomes too large/small due to inserts/deletes,
the PMA is rebalanced. A new array is allocated, and the elements are
spread out evenly over the new array (too minimize element density).

Each leaf in the static CO B-tree used is a pointer into a $\Theta(\log
N)$-size region in the packed-memory array. 

\subsubsection{Operations}

\Search{} works as follows. Search the static tree for $k$ until
a leaf node is reached. Follow the leaf node pointer into the PMA.
Scan right in the PMA until either $k$ is located, or some element
$> k$ is reached. Do not acquire any locks during the search.

\Insert{} works as follows. Use \Search{} to locate the $\Theta(\log N)$-region
of the PMA which (would) contain $k$. Lock the region, and scan right to find
the appropriate slot $s$ to place $k$ into. $s$ is defined such that slot $s -
1$ contains the largest key smaller than $k$. If scanning forward requires
going into the next $\Theta(\log N)$-region, lock the next region and release
the current region (\textit{hand-over-hand} locking technique). Once $s$
is located, either $s$ is free, in which case just insert $k$, or $s$ is not free,
in which case a \textit{rebalance} operation is required.

To do a rebalance operation, we scan right (locking as necessary) until the
region from $s$ to the current position $s'$ is not \textit{too dense}. The
notion of density is defined in detail in \cite{BenderFiGi05}. The region
$[s, s']$ is then rebalanced as follows. Start from the rightmost element,
and move elements to the right (maintaining order) such that no suffix of
$[s, s']$ is too dense. This rebalancing creates room for $k$; we insert $k$
and then release the locks.

\subsubsection{Complexity}

% TODO: can somebody check this time-bound? seems kind of bad
Sequential search time-bounds are not presented in the paper,
but presumably since search is a static CO tree lookup followed
by a scan, the time bound is $O(\log_{B}{N} + {\log_2{N}}/{B} + 1)$.

Sequential insert is $O(\log_{B}{N} + \log_2^2{N}/{B} + 1)$. The proof
is based off an accounting argument which places $\Theta(\log_2^2{m}/B)$
dollars in each array slot. The proof is not presented in any detail in the
paper, but rather it is a slightly-modified version of the proof presented
in \cite{Katriel02}. 

\subsubsection{Lock-free Variant}
The lock-free variant is based on the PMA CO tree, except the locks acquired
during the PMA modification are replaced by augmenting each element in the PMA
with a marker. A marker is used to denote the ongoing (concurrent) operation
of a particular cell.

In addition, four non-blocking primitives are introduced:
\begin{inparaenum}[(a)]
  \item \textit{move},
  \item \textit{cell-insertion},
  \item \textit{cell-deletion}, and
  \item \textit{read}
\end{inparaenum}. These primitives are self-explanatory, except they
all work in a non-blocking manner (they can fail at any point).

The concurrency protocol works as follows. Any of the non-blocking primitives
are initiated by denoting the cell in question with the marker. If a process
ever encounters a cell with a marker, it drops what it is doing and first helps
to complete the operation. When the originating process gets context-switched
back in, any SCs will fail, and then the process can check to see if the
operation was already completed. The implementation of the non-blocking
primitives is spelled out in more detail in \cite{BenderFiGi05}, but the basic
idea is using LL/SC to detect when concurrent operations are on-going.

\subsubsection{Lock-free Variant Operations}

\Search{} is un-modified from the lock-based version.

\Insert{} is modified as follows. Cell $s$ is located as before. A
\textit{cell-insertion} operation is performed on $s$. If it succeeds, we are
done. If it fails, then a rebalance is necessary. Rebalance proceeds as before,
scanning right to find the region $[s, s']$. However, whenever a non-empty cell
is encountered, a load-link (LL) is performed on the cell.  Once the region is
found, the rebalancing-to-the-right is performed as before.  However, a
store-conditional (SC) is performed on the destination of each cell.  If any SC
fails, then this means concurrent modification had occurred, in which case the
algorithm restarts from the beginning.

\section{Conclusion}

%\end{document}  % This is where a 'short' article might terminate

%ACKNOWLEDGMENTS are optional
\section{Acknowledgments}
Thanks Karger

%
% The following two commands are all you need in the
% initial runs of your .tex file to
% produce the bibliography for the citations in your paper.
\bibliographystyle{abbrv}
\bibliography{citations}  % sigproc.bib is the name of the Bibliography in this case
% You must have a proper ".bib" file
%  and remember to run:
% latex bibtex latex latex
% to resolve all references
%
% ACM needs 'a single self-contained file'!
%
%APPENDICES are optional
%\balancecolumns
%\appendix
%%Appendix A
%\section{Headings in Appendices}
%The rules about hierarchical headings discussed above for
%the body of the article are different in the appendices.
%In the \textbf{appendix} environment, the command
%\textbf{section} is used to
%indicate the start of each Appendix, with alphabetic order
%designation (i.e. the first is A, the second B, etc.) and
%a title (if you include one).  So, if you need
%hierarchical structure
%\textit{within} an Appendix, start with \textbf{subsection} as the
%highest level. Here is an outline of the body of this
%document in Appendix-appropriate form:
%\subsection{Introduction}
%\subsection{The Body of the Paper}
%\subsubsection{Type Changes and  Special Characters}
%\subsubsection{Math Equations}
%\paragraph{Inline (In-text) Equations}
%\paragraph{Display Equations}
%\subsubsection{Citations}
%\subsubsection{Tables}
%\subsubsection{Figures}
%\subsubsection{Theorem-like Constructs}
%\subsubsection*{A Caveat for the \TeX\ Expert}
%\subsection{Conclusions}
%\subsection{Acknowledgments}
%\subsection{Additional Authors}
%This section is inserted by \LaTeX; you do not insert it.
%You just add the names and information in the
%\texttt{{\char'134}additionalauthors} command at the start
%of the document.
%\subsection{References}
%Generated by bibtex from your ~.bib file.  Run latex,
%then bibtex, then latex twice (to resolve references)
%to create the ~.bbl file.  Insert that ~.bbl file into
%the .tex source file and comment out
%the command \texttt{{\char'134}thebibliography}.
%% This next section command marks the start of
%% Appendix B, and does not continue the present hierarchy
%\section{More Help for the Hardy}
%The acm\_proc\_article-sp document class file itself is chock-full of succinct
%and helpful comments.  If you consider yourself a moderately
%experienced to expert user of \LaTeX, you may find reading
%it useful but please remember not to change it.
\balancecolumns
% That's all folks!
\end{document}
